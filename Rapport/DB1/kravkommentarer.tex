1. Logge på. Ansatte får tilgang til kalendersystemet ved å logge seg på kalenderklienten
med brukernavn og passord.


2. Legge inn avtale. Ansatte skal kunne legge inn avtaler i kalenderene sine. En avtale
legges inn på avtaledato med et start- og sluttidspunkt, samt en kort beskrivelse av
avtalen ("Bil på verksted") og eventuelt sted for avtalen ("Strandveien Auto").
3. Slette avtale. Ansatte skal kunne slette en avtale som ligger i en av kalenderene sine.


4. Endre avtale. Ansatte skal kunne endre på en avtale som ligger i en av kalenderene
sine. Alle feltene kan endres.



5. Kalle inn til møte. En ansatt skal kunne kalle andre ansatte og grupper i Firma X inn til
et møte. Den som kaller inn til møte kalles en møteleder. En ansatt skal kunne kalle
inn til møte på samme måte som han/hun legger ny avtale inn i en kalender. I tillegg til
feltene for en vanlig avtale, inneholder også møteinnkallingen en liste over innkalte
møtedeltakere.


6. Motta møteinnkalling. Når en ansatt mottar innkalling til et møte, kan han/hun svare
'Godta' eller 'Forkast'. Ved å svare 'Godta', legges møteinnkallingen inn som en avtale
i den innkalte ansattes kalender. Om den ansatte svarer 'Avslag', sendes svar tilbake til
møtelederen om at innkallingen ikke er godtatt. Møteleder kan da velge å finne et nytt
tidspunkt, avlyse møtet (se under) eller fjerne deltakeren fra innkallingslista.


7. Endre møteinnkalling. Møteleder kan endre tidspunkt på en møteinnkalling. Det
sendes da beskjed ut til alle møtedeltakerne, som kan svare 'Godta' eller 'Forkast'. Ved
å svare 'Godta', endres avtalen i den innkalte ansattes kalender. Ved å svare 'Forkast'
sendes beskjed ut til alle innkalte møtedeltakere. Møteleder kan da velge å finne et
nytt tidspunkt eller å avlyse møtet (se under).


8. Avlyse møte. Møteleder kan avlyse et møte. Det sendes da beskjed til alle
møtedeltakerne, og systemet sletter møtet i deltakernes personlige kalender.


9. Melde avbud for møte. En ansatt kan melde avbud på en møteinnkalling ved å slette
avtalen i sin personlige kalender. Når en ansatt melder avbud, sendes melding til alle
de andre møtedeltakerne. Møteleder kan da velge om møtet skal avlyses eller om
han/hun skal endre tidspunkt på møtet.


10. Reservere møterom. I stedet for å skrive inn sted for en avtale eller et møte, skal
brukeren kunne reservere møterom. Kalendertjeneren skal lage en liste med
tilgjengelige møterom (tilgjengelig betyr ikke reserverte) i tidsperioden for
avtalen/møtet. Brukeren kan da velge møterom fra denne listen. Om en avtale med
reservert møterom slettes, skal reservasjonen slettes på kalendertjeneren. Det samme
gjelder for møter som avlyses.


11. Visning. Kalenderklienten skal vise en ukekalender der alle avtaler og møter i den
ansattes personlige kalender vises. Det skal være enkelt å bla mellom ukene.


12. Spore møteinnkallinger. Kalenderklienten skal indikere i ukekalenderen om a) en
møteinnkalling venter på svar fra en eller flere deltakere, b) en eller flere
møtedeltakere har avslått møteinnkalling, eller c) om alle innkalte har godtatt
møteinnkallingen.


13. Vis flere kalendre. Det skal være mulig å vise andre ansattes avtaler sammen med sine
egne i kalenderklienten.


14. Alarm. Det skal være mulig for hver ansatt å konfigurere enhver avtale slik at avtalen
genererer en alarm en gitt tid før møtet.

