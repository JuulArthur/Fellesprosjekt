Vi skal tilby en forbindelsesorientert nettverksløsning, og dette skal vi gjøre ved hjelp av A1 og A2. A2 eksisterer allerede og er et forbindelsesløst nett. All kontakt mellom klienten og serveren for kalenderapplikasjonen vår skal gå igjennom A1, som deretter tar kontakt med A2 (socketen). Tilkoblingen mot A1 skal være pålitelig og tapsfri, akkurat som TCP. Vi skal endre på ConnectionImpl-klassen som implementerer Connection-klassen. Det er 4 sentrale metoder vi skal impelentere. Disse er accept(), connect(address, port), close(), send() og receive(). Om vi skulle ønske, og ha behov, kan vi også implementere isValid(packet). A1 skal kommunisere med A2 gjennom metodene send, receive og cancel\_receive som ligger i A2.

For å kunne opprette en pålitelig tilkopling igjennom A2, må vi selv opprette koplinger for data som skal sendes. Når dette er gjort kan vi sikre tapsfri overføring. I sekvensdiagrammene vi har laget, viser vi hvordan serveren og klienten kan abstrahere bort mange funksjoner som alltid må kjøres, og dermed bare konsentrere seg om det essensielle. Vi viser spesifikt tilfellene Connect, Send og Close.

Når vi oppretter en connection skal det utføres en three-way-handshake. Først sender vi en SYN fra klienten til servern, i respons får vi en SYN-ACK tilbake, og tilslutt sender klienten en ACK til servern og det har blitt dannet en tilkobling.

Hver gang en pakke blir sendt så sender man med et sekvensnummer, deretter får man tilbake en ACK og hva mottakeren forventer at neste sekvensnummer skal være, dermed vet man at pakken kom fram. 
Når en tilkobling skal bli lukket vil det utføres en three-way-handshake her også. Dette foregår ved at klienten sender en FIN mot servern, som vil svare med ACK (servern har fått pakken) og en FIN selv. Klienten får dette tilbake og svarer da med en ACK selv, og tilkoblingen blir avsluttet.
For å kunne lage en god kommunikasjon, er det fint å vite hva som finnes i A2. Dette kan aksesseres ved å bruke Admin-klassen. Vi kan her skrive ut logger fra A2 og gjøre innstillinger. Med disse verktøyene kan vi lage en pålitelig og god tilkopling.