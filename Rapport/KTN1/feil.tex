\textbf{Beskrivelsen av A2 tilsier at det kan oppstå fire forskjellige typer feil med tanke på pakkeoverføring. Vi skal implementere A1, som skal behandle, og rette opp disse feilene.}

\subsection{Package lost}
Hvis senderen ikke mottar en ACK før den når timeout-tiden vil en pakke bli regnet som tapt.
Pakker med lavere sekvensnummer blir regnet som tapt hvis sekvensnummeret vi mottar er høyere enn forventet. 
Hvis dette skjer vil samme pakke bli sendt på nytt, helt til senderen mottar en ACK før timeout-tiden, og får forventet sekvensnummer tilbake.
Metoden sendDataPacketWithRetransmit() i klassen AbstractConnection gjør akkurat dette for oss. 

\subsection{Package delayed}
Hvis mottakeren sender ACK to ganger for samme pakke, betyr det at pakken har blitt sendt to ganger fordi den første ikke kom fram raskt nok.
Dette har ikke noe å si for senderen, han vil bare ignorere den andre ACKen. Mottakeren vil ignorere en pakke han allerede har fått. 
 
\subsection{Package has errors}
Vi må her sjekke om pakken mottatt er samme pakke som ble sendt og inneholder det samme sekvensnummeret. Vi kan her implementere isValid()-funksjonen for å gjøre dette. Her brukers checksumen som blir sendt med pakken for å sjekke om pakken inneholder feil.
 
\subsection{Ghost package}
Her er det snakk om en pakke ingen venter, som bare oppstår. Vi kan sjekke sekvensnummer, og bruke isValid()-funksjonen for å eliminere dette. 