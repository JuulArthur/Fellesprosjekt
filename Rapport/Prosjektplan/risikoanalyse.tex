\subsection{Risikoscenarioer}
\begin{center}
\label{tab:risikoscnarioer}
\begin{longtable}{| l | L{4cm} | L{4cm} | L{4cm} |}
\hline
ID & Situasjon & Forebyggende tiltak & Løsningsplan \\
\hline
1 & En eller flere personer er ikke interessert i å bidra, kommer ikke til planlagte møter, vanskelig å få kontakt med. & Være inkluderende på møter. Prøve å variere på arbeidsoppgaver så ingen går lei. Forståelse for at kompetansen i gruppen kan variere veldig. & SCRUM-master tar direkte kontakt med personen,bytte oppgaver hvis personen er misfornøyd, evt ta kontakt med und.ass/faglærer og avtale møte som alle i gruppen må delta på.\\
\hline
2 & Tidsestemering slår feil. En eller flere oppgaver er mye mer tidskrevende enn antatt. & Ha møte hver morgen der vi forteller om hva vi har gjort og eventuelt om noe var mer tidkrevende enn først antatt. Møter med und.ass hvor vi får tips om hvilke oppgaver vi burde sette av mest tid til. & Sette av mer tid til oppgaven. Kompensere ved å ha flere arbeidstimer (vurdere overtid), eventuelt omallokere ressursene våres. Droppe noen ikke-essensielle krav for oppgaven. Diskutere på møte hvorfor tidsestemeringen gikk galt, og lære av dette. \\
\hline
3 & To eller flere gruppemedlemmer misliker hverandre sterkt, og nekter å samarbeide. Skader produktiviteten og gruppemoralen. & God dialog innad i gruppen. Personer som ikke samarbeider bra blir satt til forskjellige arbeidsoppgaver. & Først prøve en gruppesamtale, hvor SCRUM-master prøver å løse konfliktene.
Samtale med und.ass, foreslå bytting av grupper hvis nødvendig. \\
\hline
4 & En eller flere personer tar på seg en for stor oppgave. Blir veldig skeiv arbeidsfordeling, noen gjør mye mer enn andre. & Snakke om hva alle har gjort siden sist på hvert møte. Inkludere alle i planleggingen, og passe på at ingen tar oppgaver de ikke har kompetanse til å håndtere. Sørge for at oppgavene er jevnt fordelt utover alle gruppemedlemmer. & Omrokkere på arbeidsoppgaver og sørge for bedre dialog. Ta det opp på møtet. \\
\hline
5 & Alle i gruppen har \underline{ikke} aktive arbeidsoppgaver til en hver tid. & Kartlegge kompetansen veldig tidlig, og alltid ha dette i bakhodet ved delegering av oppgaver. Fortelle om arbeidsoppgavene våres for dagen på daglige møter. Være aktive på IRC. Sørge for at vi har en gruppefølelse og at vi har lyst til å lykkes sammen. & Sette personer som ikke har arbeidsoppgaver over til nye arbeidsoppgaver, eventuelt hjelpe andre som fortsatt har flere oppgaver igjen. \\
\hline
6 & Kompetansen til de ulike medlemmene blir ikke brukt på en effektiv måte. & Da har vi ikke gjort en god nok jobb i forhold til punkt 5 og må ta det opp på morgenmøtet for å løse problemet. & . \\
\hline
7 & Uforventet fravær fra ett/ flere gruppemedlemmer. & Vi må være aktive i å informere om noe forekommer. & Omallokere arbeidsoppgaver og prioritere arbeidsoppgaver som er viktigere. Sette av mer tid til oppgavene. \\
\hline
\caption{Tabell med oversikt over risikoanalysen vi har foretatt}
\end{longtable}
\end{center}

\subsection{Scenariovurdering}
\begin{table}[H]
\centering
\label{tab:scenariovurdering}
\begin{tabular}{| l | l | l |}
\hline
ID & Sannsynlighet & Alvorlighetsgrad \\
\hline
1 & Liten & Alvorlig \\
\hline
2 & Svært stor & Liten \\
\hline
3 & Moderat & Alvorlig \\
\hline
4 & Stor & Moderat \\
\hline
5 & Stor & Liten \\
\hline
6 & Moderat & Moderat \\
\hline
7 & Stor & Moderat \\
\hline
\end{tabular}
\caption{Tabell med oversikt over scenariovurdering}
\end{table}