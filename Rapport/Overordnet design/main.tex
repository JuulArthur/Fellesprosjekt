\documentclass[titlepage]{article}
\usepackage[norsk]{babel}
\usepackage[utf8]{inputenc}
%\usepackage[latin1]{inputenc}
\usepackage{graphicx}
\usepackage{float}
\usepackage{parskip}
\usepackage{array}
\newcolumntype{L}[1]{>{\raggedright\let\newline\\\arraybackslash\hspace{0pt}}m{#1}}
\newcolumntype{C}[1]{>{\centering\let\newline\\\arraybackslash\hspace{0pt}}m{#1}}
\newcolumntype{R}[1]{>{\raggedleft\let\newline\\\arraybackslash\hspace{0pt}}m{#1}}
\usepackage{longtable}

\author{Gruppe 38}
\title{Overordnet design}
\date{\today}

\begin{document}

\maketitle

\begin{abstract}
Overall design gives value-added synergy combined with real-time monetizing and the posibility for scalability if big data. All in an agile way.
\end{abstract}

\tableofcontents


\newpage
\section{Innledning}
Vi fikk oppgave i å lage et kalendersystem for firma X der de kan benytte den til å organisere møter og andre avtaler. Etter at vi har kommet med designet så måtte vi se hvor bra designet vårt er. For å se hvordan produktet som vi skal lage kan forbedres og hvor brukervennlig den egentlig er, så fikk vi en testperson til å gjennomføre en prototypetesting for oss der vi benyttet en papirprototype. En papirprototype er en primitiv og enkel papirutgave av det  ferdige produktet. Vi skal kunne simulere hva produktet skal gjøre og dermed se eventuelle problemer og uklarheter hos brukeren. Ved hjelp av en papirprototype så kan vi observere hvordan brukeren interaktivere med produktet og observere eventuelle måter vi kan forbedre produktet på. 

\newpage
\section{Scenario 1}
\subsection{Use Case}
\begin{figure}[H]
\label{fig:uc1}
\includegraphics[width=400px]{ucs1.png}
\caption{Use Case-diagram for scenario 1}
\end{figure}

\subsection{Sekvensdiagram}
\begin{figure}[H]
\label{fig:sek1}
\includegraphics[width=400px]{sekvens1.png}
\caption{Use Case-diagram for scenario 1}
\end{figure}

\newpage
\section{Scenario 2}
\subsection{Use Case}
\begin{figure}[H]
\label{fig:uc2}
\includegraphics[width=400px]{ucs2.png}
\caption{Use Case-diagram for scenario 2}
\end{figure}

\subsection{Tekstlig Use Case}
\begin{table}[H]
\centering
\label{tab:tuc2}
\begin{tabular}{| l | L{4in} |}
\hline
Use Case & Scenario 2 \\
\hline
Aktør & Møteleder og inviterte ansatte \\
\hline
Trigger & Møteleder inviterer ansatte til et møte \\
\hline
Pre-betingelser & Møteleder har laget møtet \\
\hline
Post-betingelser & Ansatte er invitert til møtet, og mottar melding om dette når de logger seg på systemet. \\
\hline
Normal hendelsesflyt & 
\begin{minipage}{4in}
\vskip 4pt
\begin{itemize}
\item Møteleder opprettet møtet
\item Møteleder inviterer ansatte til møtet
\item Melding blir sendt til de inviterte, som de mottar neste gang de logger seg på
\end{itemize}
\vskip 4pt
\end{minipage}
 \\
\hline
Variasjoner & \\
\hline
Relatert informasjon & De inviterte medlemmene ser meldingen neste gang de logger seg på systemet. \\
\hline
\end{tabular}
\caption{Tekslig Use Case-diagrame}
\end{table}

\subsection{Sekvensdiagram}
\begin{figure}[H]
\label{fig:sek2}
\includegraphics[width=400px]{sekvens2.png}
\caption{Use Case-diagram for scenario 2}
\end{figure}

\newpage
\section{Scenario 3}
\subsection{Use Case}
\begin{figure}[H]
\label{fig:uc3}
\includegraphics[width=400px]{ucs3.png}
\caption{Use Case-diagram for scenario 3}
\end{figure}

\subsection{Tekstlig Use Case}
\begin{table}[H]
\centering
\label{tab:tuc3}
\begin{tabular}{| l | L{4in} |}
\hline
Use Case & Scenario 3 \\
\hline
Aktør & Møteleder og inviterte ansatte \\
\hline
Trigger & Møteleder inkaller til møte \\
\hline
Pre-betingelser & Brukerene må være logget inn og invitert til samme møte. \\
\hline
Post-betingelser & Møtet blir satt hvor brukerene som har godtatt er invitert og de som har avlyst blir slettet fra møteinkallelsen. \\
\hline
Normal hendelsesflyt & 
\begin{minipage}{4in}
\vskip 4pt
\begin{itemize}
\item Møteleder inkaller til møte
\item Inviterte ansatte mottar møteinkallelse
\item Noen akspeterer møteinkalleslsen og noen avslår møteinkallelsen
\end{itemize}
\vskip 4pt
\end{minipage}
 \\
\hline
Variasjoner & 
\begin{minipage}{4in}
\vskip 4pt
\begin{itemize}
\item Møte blir planlagt på nytt
\item De som avslår møteinkallelsen blir slettet fra møteinkallelsen
\end{itemize}
\vskip 4pt
\end{minipage}
\\
\hline
Relatert informasjon & Alle ansatte som er invitert får beskjed om endringer som blir gjort i møteinkallelsen. \\
\hline
\end{tabular}
\caption{Tekslig Use Case-diagrame}
\end{table}

\subsection{Sekvensdiagram}
\begin{figure}[H]
\label{fig:sek3}
\includegraphics[width=400px]{sekvens3.png}
\caption{Use Case-diagram for scenario 3}
\end{figure}

\newpage
\section{Scenario 4}
\subsection{Tekstlig Use Case}
\begin{table}[H]
\centering
\label{tab:tuc4}
\begin{tabular}{| l | L{4in} |}
\hline
Use Case & Scenario 3 \\
\hline
Aktør & Møteleder og inviterte ansatte \\
\hline
Trigger & Møtelederen avlyser møtet \\
\hline
Pre-betingelser & Møtelederen har laget møtet og invitert andre ansatt. Noen ansatt må også ha godtat møtet. \\
\hline
Post-betingelser & Møten slettes og det gis bedskjed til de involverte ansatt. \\
\hline
Normal hendelsesflyt & 
\begin{minipage}{4in}
\vskip 4pt
\begin{itemize}
\item Møtelederen avlyser møtet.
\item Møtet blir slettet fra Møtelederens personlige kalender.
\item En melding blir sendt til alle inviterte medlemer.
\end{itemize}
\vskip 4pt
\end{minipage}
 \\
\hline
Variasjoner & \\
\hline
Relatert informasjon & De inviterte medlemer ser meldingen neste gang de logger seg på systemet. \\
\hline
\end{tabular}
\caption{Tekslig Use Case-diagrame}
\end{table}

\subsection{Sekvensdiagram}
\begin{figure}[H]
\label{fig:sek4}
\includegraphics[width=400px]{sekvens4.png}
\caption{Use Case-diagram for scenario 4}
\end{figure}

\newpage
\section{Klassediagram}
\includegraphics[scale=0.75]{Klassediagram.png}

\newpage
\listoftables

\newpage
\listoffigures

\newpage
\begin{thebibliography}{9}

\bibitem{fpkomp}
	Kompendium til fellesprosjektet,
	\emph{it's learning-gruppa til faget}
\end{thebibliography}

\end{document}
